\documentclass[mathserif]{beamer}
\usepackage{listings}
\lstset{language=Java}

\usetheme{Boadilla}
\usecolortheme{seahorse}

\setbeamertemplate{enumerate items}{(\arabic{enumi})}
\setbeamertemplate{itemize items}[circle]
\setbeamertemplate{navigation symbols}{}

\title{Vorbereitung auf das Programmierdiktat}
\author{Michael Strassberger\\michael.strassberger@uni-hamburg.de\\3strassb@informatik.uni-hamburg.de}

\begin{document}

\begin{frame}
\titlepage
\end{frame}

\begin{frame}{Fahrplan}
\tableofcontents
\end{frame}

\section{Syntax einer Klasse}

\begin{frame}[fragile]
\frametitle{Klassen und Interfaces definieren (Begriffe „Rumpf“ und „Kopf“ sollten auch bekannt sein)}
\pause
    \begin{lstlisting}
class Buch // Kopf
{
    // Rumpf
}
    \end{lstlisting}
\pause
    \begin{lstlisting}
interface Medium // Kopf
{
    // Rumpf
}
    \end{lstlisting}
\end{frame}

\begin{frame}[fragile]
\frametitle{Implementationsbeziehungen zwischen einer Klasse und einem Interface definieren}
\pause
    \begin{lstlisting}
class Buch implements Medium 
{

}
    \end{lstlisting}
\end{frame}

\begin{frame}[fragile]
\frametitle{Variablen deklarieren, initialisieren, zuweisen}
\pause
    \begin{lstlisting}
int nummer;
    \end{lstlisting}
\pause
    \begin{lstlisting}
int nummer = 42;
    \end{lstlisting}
\pause
    \begin{lstlisting}
    nummer = 43;
    \end{lstlisting}
\end{frame}

\begin{frame}[fragile]
\frametitle{(Zustands-) Felder (alias Exemplarvariablen) deklarieren}
    \begin{lstlisting}
class Buch implements Medium 
{
    \end{lstlisting}
\pause
    \begin{lstlisting}
    int _seiten;
}
    \end{lstlisting}
\end{frame}

\begin{frame}[fragile]
\frametitle{Zugriffsmodifikatoren verwenden k\"onnen (public/private)}
    \begin{lstlisting}
class Buch implements Medium 
{
    \end{lstlisting}
\pause
    \begin{lstlisting}
    private int _seiten;
}
    \end{lstlisting}
\end{frame}

\begin{frame}[fragile]
\frametitle{Konstruktoren definieren}
    \begin{lstlisting}
class Buch implements Medium 
{
    private int _seiten;
    \end{lstlisting}
\pause
    \begin{lstlisting}
    public Buch()
    {
        _seiten = 42;
    }
}
    \end{lstlisting}
\end{frame}

\begin{frame}[fragile]
\frametitle{Werte von Methoden zurückgeben lassen (Schlüsselwort „return“)}
    \begin{lstlisting}
class Buch implements Medium 
{
    private int _seiten;
    \end{lstlisting}
\pause
    \begin{lstlisting}
    public int gibSeitenAnzahl()
    {
        return _seiten;
    }
}
    \end{lstlisting}
\end{frame}

\begin{frame}[fragile]
\frametitle{Klassenvariablen deklarieren, initialisieren}
    \begin{lstlisting}
class Buch implements Medium
{
    \end{lstlisting}
    \pause
    \begin{lstlisting}
    private static int anzahlBuecher;
    \end{lstlisting}
    \pause
    \begin{lstlisting}
    static
    {
        anzahlBuecher = 0;
    }
}
    \end{lstlisting}
\end{frame}

\begin{frame}[fragile]
\frametitle{Private und \"offentliche Methoden/Operationen definieren, sowohl als Exemplar- als auch als Klassenmethoden (Begriffe „Rumpf“ und „Kopf“ sollten auch bekannt sein)}
 \begin{lstlisting}
    public int gibSeitenAnzahl()
    {
        return _seiten;
    }
    \end{lstlisting}
    \pause
     \begin{lstlisting}
    private int gibSeitenAnzahlGeheim()
    {
        return _seiten;
    }
    \end{lstlisting}
\end{frame}

\begin{frame}[fragile]
    \frametitle{Klassenmethoden}
    \begin{lstlisting}
class Buch implements Medium
{
    private static int anzahlBuecher;
    static
    {
        anzahlBuecher = 0;
    }
    \end{lstlisting}
    \pause
    \begin{lstlisting}
    public static int gibAnzahlBuecher()
    {
        return anzahlBuecher;
    }
}
    \end{lstlisting}
\end{frame}

\begin{frame}[fragile]
    \frametitle{Recap: Rumpf von Klassen und Methoden}
    \begin{lstlisting}
class Buch implements Medium // Klassen Kopf
{
    // Klassen Rumpf

    public Buch() // Methoden Kopf
    {
        // Methoden Rumpf
    }
}
    \end{lstlisting}
\end{frame}

\begin{frame}[fragile]
\frametitle{Parameter deklarieren und verwenden}
    \begin{lstlisting}
class Buch implements Medium
{
    private int _seiten;
    \end{lstlisting}
    \pause
    \begin{lstlisting}
    public void setzeSeitenAnzahl(int seiten)
    {
    \end{lstlisting}
    \pause
    \begin{lstlisting}
        _seiten = seiten;
    }
}
    \end{lstlisting}
\end{frame}

\begin{frame}[fragile]
    \frametitle{Main Methode}
    \begin{lstlisting}
public class SE2Tutorium
{
    \end{lstlisting}
    \pause
    \begin{lstlisting}
    public static void main(String[] args)
    {

    }
}
    \end{lstlisting}
\end{frame}

\begin{frame}[fragile]
\frametitle{Objekte erzeugen}
    \begin{lstlisting}
public class SE2Tutorium
{
    public static void main(String[] args)
    {
    \end{lstlisting}
    \pause
    \begin{lstlisting}
        Buch bluej = new Buch();
    }
}
    \end{lstlisting}
\end{frame}

\begin{frame}[fragile]
\frametitle{Umgang mit Referenzvariablen und der Punktnotation}
    \begin{lstlisting}
public class SE2Tutorium
{
    public static void main(String[] args)
    {
        Buch bluej = new Buch();
    \end{lstlisting}
    \pause
    \begin{lstlisting}
        int bluejSeiten = bluej.getSeitenAnzahl();
    \end{lstlisting}
    \pause
    \begin{lstlisting}
        bluej.setzeSeitenAnzahl(bluejSeiten + 1);
    }
}
    \end{lstlisting}
\end{frame}

\begin{frame}
    \center{\huge{Noch Fragen zu Klassen und Methoden R\"umpfen}}
\end{frame}

\section{Datenstrukturen}

\begin{frame}[fragile]
\frametitle{Umgang mit Basisdatentypen}
    \begin{enumerate}
        \item byte
        \item short
        \item int
        \item long
        \pause
        \item float
        \item double
        \pause
        \item char
    \end{enumerate}
\end{frame}

\begin{frame}[fragile]
\frametitle{Umgang mit Arrays (deklarieren, initialisieren, schreibender und lesender Zugriff)}
\pause
    \begin{lstlisting}
int[] zahlenkette
    \end{lstlisting}
\pause
    \begin{lstlisting}
zanlenkette = new int[42]
    \end{lstlisting}
\end{frame}

\begin{frame}[fragile]
\frametitle{Umgang mit Listen und Mengen}
\pause
    \begin{lstlisting}
List<Integer> zahlenListe;
    \end{lstlisting}
\pause
    \begin{lstlisting}
zahlenListe = new ArrayList();
    \end{lstlisting}
\pause
    \begin{lstlisting}
zahlenListe.add(100);
    \end{lstlisting}
\pause
    \begin{lstlisting}
zahlenListe.remove(0);
    \end{lstlisting}
\end{frame}

\section{Kontrollstrukturen}

\begin{frame}[fragile]
\frametitle{Switch-Anweisungen schreiben}
\pause
    \begin{lstlisting}
switch (zahl) {
    case 10:
        // do things
        break;
    default:
        break
}

    \end{lstlisting}
\end{frame}

\section{Fragen}
\begin{frame}
    \center{\huge{Gibt es noch Fragen?}}
    \pause
    \center{\huge{Dann viel Erfolg beim Diktat!}}
\end{frame}
\end{document}
